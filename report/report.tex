\documentclass{article}


% if you need to pass options to natbib, use, e.g.:
%     \PassOptionsToPackage{numbers, compress}{natbib}
% before loading neurips_2023


% ready for submission
\usepackage[preprint]{neurips_2023}


% to compile a preprint version, e.g., for submission to arXiv, add add the
% [preprint] option:
%     \usepackage[preprint]{neurips_2023}


% to compile a camera-ready version, add the [final] option, e.g.:
%     \usepackage[final]{neurips_2023}


% to avoid loading the natbib package, add option nonatbib:
%    \usepackage[nonatbib]{neurips_2023}


\usepackage[utf8]{inputenc} % allow utf-8 input
\usepackage[T1]{fontenc}    % use 8-bit T1 fonts
\usepackage{hyperref}       % hyperlinks
\usepackage{url}            % simple URL typesetting
\usepackage{booktabs}       % professional-quality tables
\usepackage{amsfonts}       % blackboard math symbols
\usepackage{nicefrac}       % compact symbols for 1/2, etc.
\usepackage{microtype}      % microtypography
\usepackage{xcolor}         % colors


\title{Finding the causes of failure: Neuralwave 2024}


% The \author macro works with any number of authors. There are two commands
% used to separate the names and addresses of multiple authors: \And and \AND.
%
% Using \And between authors leaves it to LaTeX to determine where to break the
% lines. Using \AND forces a line break at that point. So, if LaTeX puts 3 of 4
% authors names on the first line, and the last on the second line, try using
% \AND instead of \And before the third author name.


\author{
	Mark Sobolev\\
	Faculty of Informatics\\
	Università della Svizzera Italiana\\
	Lugano, Switzerland \\
	\\
	% examples of more authors
	% \And
	% Coauthor \\
	% Affiliation \\
	% Address \\
	% \texttt{email} \\
	% \AND
	% Coauthor \\
	% Affiliation \\
	% Address \\
	% \texttt{email} \\
	% \And
	% Coauthor \\
	% Affiliation \\
	% Address \\
	% \texttt{email} \\
	% \And
	% Coauthor \\
	% Affiliation \\
	% Address \\
	% \texttt{email} \\
}


\begin{document}
	
	
	\maketitle
	

	\begin{abstract}
		We proposed a solution to a causal discovery in a multiple dataset context setting with a prior structural knowledge. Given the different measurements set, we produced an directed graph of causal influences and a ranking of an influencing variables. In addition we implemented a way to precisely constraint our causal model to reflect the changed mode. We developed a complex evaluation metric set for the unsupervised settings. Finally we designed a way to combine different approaches to causal discovery to produce the most reliable approach to generalised causal discovery setting.
	\end{abstract}
	
	
	\section{Objective}
	
	We've been tackling a problem, where we were given several datasets with different latent context, for those we needed to provide the directed acyclic graph, which could be interpreted casually. Namely the datasets were structured as a set of columns, which were grouped in a several blocks, with an assumption of locality between elements of a block. We needed to discover various causal related structures about the given data:
	
	\begin{enumerate}
		\item The underlying causal graph
		\item The root causes of the observed change
		\item Visualisation of the graph, the root causes and their influence
	\end{enumerate}
 
	\begin{center}
		\url{http://www.neurips.cc/}
	\end{center}


	The documentation for \verb+natbib+ may be found at
	\begin{center}
		\url{http://mirrors.ctan.org/macros/latex/contrib/natbib/natnotes.pdf}
	\end{center}
	Of note is the command \verb+\citet+, which produces citations appropriate for
	use in inline text.  For example,
	\begin{verbatim}
		\citet{hasselmo} investigated\dots
	\end{verbatim}
	produces
	\begin{quote}
		Hasselmo, et al.\ (1995) investigated\dots
	\end{quote}
	
	
	If you wish to load the \verb+natbib+ package with options, you may add the
	following before loading the \verb+neurips_2023+ package:
	\begin{verbatim}
		\PassOptionsToPackage{options}{natbib}
	\end{verbatim}
	
	
	If \verb+natbib+ clashes with another package you load, you can add the optional
	argument \verb+nonatbib+ when loading the style file:
	\begin{verbatim}
		\usepackage[nonatbib]{neurips_2023}
	\end{verbatim}
	
	
	As submission is double blind, refer to your own published work in the third
	person. That is, use ``In the previous work of Jones et al.\ [4],'' not ``In our
	previous work [4].'' If you cite your other papers that are not widely available
	(e.g., a journal paper under review), use anonymous author names in the
	citation, e.g., an author of the form ``A.\ Anonymous'' and include a copy of the anonymized paper in the supplementary material.
	
	
	\subsection{Footnotes}
	
	
	Footnotes should be used sparingly.  If you do require a footnote, indicate
	footnotes with a number\footnote{Sample of the first footnote.} in the
	text. Place the footnotes at the bottom of the page on which they appear.
	Precede the footnote with a horizontal rule of 2~inches (12~picas).
	
	
	Note that footnotes are properly typeset \emph{after} punctuation
	marks.\footnote{As in this example.}
	
	
	\subsection{Figures}
	
	
	\begin{figure}
		\centering
		\fbox{\rule[-.5cm]{0cm}{4cm} \rule[-.5cm]{4cm}{0cm}}
		\caption{Sample figure caption.}
	\end{figure}
	
	
	All artwork must be neat, clean, and legible. Lines should be dark enough for
	purposes of reproduction. The figure number and caption always appear after the
	figure. Place one line space before the figure caption and one line space after
	the figure. The figure caption should be lower case (except for first word and
	proper nouns); figures are numbered consecutively.
	
	
	You may use color figures.  However, it is best for the figure captions and the
	paper body to be legible if the paper is printed in either black/white or in
	color.
	
	
	\subsection{Tables}
	
	
	All tables must be centered, neat, clean and legible.  The table number and
	title always appear before the table.  See Table~\ref{sample-table}.
	
	
	Place one line space before the table title, one line space after the
	table title, and one line space after the table. The table title must
	be lower case (except for first word and proper nouns); tables are
	numbered consecutively.
	
	
	Note that publication-quality tables \emph{do not contain vertical rules.} We
	strongly suggest the use of the \verb+booktabs+ package, which allows for
	typesetting high-quality, professional tables:
	\begin{center}
		\url{https://www.ctan.org/pkg/booktabs}
	\end{center}
	This package was used to typeset Table~\ref{sample-table}.
	
	
	\begin{table}
		\caption{Sample table title}
		\label{sample-table}
		\centering
		\begin{tabular}{lll}
			\toprule
			\multicolumn{2}{c}{Part}                   \\
			\cmidrule(r){1-2}
			Name     & Description     & Size ($\mu$m) \\
			\midrule
			Dendrite & Input terminal  & $\sim$100     \\
			Axon     & Output terminal & $\sim$10      \\
			Soma     & Cell body       & up to $10^6$  \\
			\bottomrule
		\end{tabular}
	\end{table}
	
	
	If your file contains type 3 fonts or non embedded TrueType fonts, we will ask
	you to fix it.
	
	
	\subsection{Margins in \LaTeX{}}
	
	
	
	{
		\small
		
		
		[1] Alexander, J.A.\ \& Mozer, M.C.\ (1995) Template-based algorithms for
		connectionist rule extraction. In G.\ Tesauro, D.S.\ Touretzky and T.K.\ Leen
		(eds.), {\it Advances in Neural Information Processing Systems 7},
		pp.\ 609--616. Cambridge, MA: MIT Press.
		
		
		[2] Bower, J.M.\ \& Beeman, D.\ (1995) {\it The Book of GENESIS: Exploring
			Realistic Neural Models with the GEneral NEural SImulation System.}  New York:
		TELOS/Springer--Verlag.
		
		
		[3] Hasselmo, M.E., Schnell, E.\ \& Barkai, E.\ (1995) Dynamics of learning and
		recall at excitatory recurrent synapses and cholinergic modulation in rat
		hippocampal region CA3. {\it Journal of Neuroscience} {\bf 15}(7):5249-5262.
	}
	
	%%%%%%%%%%%%%%%%%%%%%%%%%%%%%%%%%%%%%%%%%%%%%%%%%%%%%%%%%%%%
	
	
\end{document}