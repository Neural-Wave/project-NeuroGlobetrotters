\documentclass{article}


% if you need to pass options to natbib, use, e.g.:
%     \PassOptionsToPackage{numbers, compress}{natbib}
% before loading neurips_2023


% ready for submission
\usepackage[preprint]{neurips_2023}
\usepackage{biblatex} %Imports biblatex package
\addbibresource{bosch.bib} %Import the bibliography file

% to compile a preprint version, e.g., for submission to arXiv, add add the
% [preprint] option:
%     \usepackage[preprint]{neurips_2023}


% to compile a camera-ready version, add the [final] option, e.g.:
%     \usepackage[final]{neurips_2023}


% to avoid loading the natbib package, add option nonatbib:
%    \usepackage[nonatbib]{neurips_2023}


\usepackage[utf8]{inputenc} % allow utf-8 input
\usepackage[T1]{fontenc}    % use 8-bit T1 fonts
\usepackage{hyperref}       % hyperlinks
\usepackage{url}            % simple URL typesetting
\usepackage{booktabs}       % professional-quality tables
\usepackage{amsfonts}       % blackboard math symbols
\usepackage{nicefrac}       % compact symbols for 1/2, etc.
\usepackage{microtype}      % microtypography
\usepackage{xcolor}         % colors


\title{Finding the causes of failure: Neuralwave 2024}


% The \author macro works with any number of authors. There are two commands
% used to separate the names and addresses of multiple authors: \And and \AND.
%
% Using \And between authors leaves it to LaTeX to determine where to break the
% lines. Using \AND forces a line break at that point. So, if LaTeX puts 3 of 4
% authors names on the first line, and the last on the second line, try using
% \AND instead of \And before the third author name.


\author{
	Mark Sobolev\\
	Faculty of Informatics\\
	Università della Svizzera Italiana\\
	Lugano, Switzerland \\
	\\
	% examples of more authors
	\And
	Marco Gabriel \\
	Faculty of Informatics \\
	Università della Svizzera Italiana\\
	Lugano, Switzerland \\
	% \\
	% \AND
	% Coauthor \\
	% Affiliation \\
	% Address \\
	% \texttt{email} \\
	% \And
	% Coauthor \\
	% Affiliation \\
	% Address \\
	% \texttt{email} \\
	% \And
	% Coauthor \\
	% Affiliation \\
	% Address \\
	% \texttt{email} \\
}


\begin{document}
	
	
	\maketitle
	

	\begin{abstract}
		We proposed a solution to a causal discovery in a multiple dataset context setting with a prior structural knowledge. Given the different measurements set, we produced an directed graph of causal influences and a ranking of an influencing variables. In addition we implemented a way to precisely constraint our causal model to reflect the changed mode. We developed a complex evaluation metric set for the unsupervised settings. Finally we designed a way to combine different approaches to causal discovery to produce the most reliable approach to generalised causal discovery setting.
	\end{abstract}
	
	
	\section{Objective} % The problem being addressed and its significance.
	
	We've been tackling a problem, where we were given several datasets with different latent context, for those we needed to provide the directed acyclic graph, which could be interpreted casually. Namely the datasets were structured as a set of columns, which were grouped in a several blocks, with an assumption of locality between elements of a block. We needed to discover various causal related structures about the given data:
	
	\begin{enumerate}
		\item The underlying causal graph
		\item The root causes of the observed change
		\item Visualisation of the graph, the root causes and their influence
	\end{enumerate}


    \section{Approach} % Methodology, algorithms, and frameworks used.

    \subsection{Methodology}

    The process is divided into three stages. First, we do not have access to any ground-truth. In the second stage, we get access to the first 15 rows of the target adjacency-matrix. In the last stage, we get additional access to the lower diagonal adjacency-matrix. We use this ground-truth as prior knowledge in the following algorithms as well as to evaluate our results.

    In every stage, we fit various algorithms to the given data and compared their results. We outline these algorithms in the following sections.

    \subsection{Algorithms}
    
    

    \subsubsection{LiNGAM}
    LiNGAM estimates structural equation models or linear causal Bayesian networks, assuming the data is non-Gaussian.
    First, to apply DirectLiNGAM\cite{lingam}, we merge the datasets while introducing a new feature to differentiate the origins. 
    Secondly, we use MultiGroupDirectLiNGAM: \cite{multilingam} on the separate datasets.
    Additionally, we try Independent component analysis (ICA-LiNGAM)\cite{ICALINGAM} which assumes the observed variables to be linear functions of mutually independent and non-Gaussian disturbance variables.

    \subsubsection{Fast Causal Inference (FCI)}
    The FCI algorithm \cite{FCI} works under both latent variables and selection bias. We use its implementation in causal-learn \cite{Causallearn}.

    \subsubsection{VARMALINGRAM}

    \subsubsection{PC}
    
    \subsubsection{Structural Agnostic Model (SAM)}
    We use the SAM\cite{SAM} implementation of the CausalDiscoveryToolbox (CDT). SAM implements an adversarial game in which separate models generate each variable based on the real observations from all other variables, on the hypothesis, that the target variable depends on those. A discriminator then attempts to distinguish between real and generated samples. Finally, a sparsity penalty forces each generator to consider only a small subset of the variables, yielding a sparse causal graph. SAM allows to alleviate the commonly made condition, that there is no hidden common cause to any pair of observed variables.
    It is shown to be orders of magnitudes slower than LiNGAM for example, but also significantly better.

    \subsubsection{LLM: o1-preview from OpenAI}
    We prompt the o1-preview LLM from OpenAI to respond with the nodes causing the increased value, given the datasets.
    
    \subsection{Frameworks}
    \subsubsection{DoWhy}
    The key feature of DoWhy \cite{DoWhy} \cite{sharma2020dowhyendtoendlibrarycausal} is that it provides algorithms for the refutation and falsification of causal graphs.
    
    
    \subsubsection{CausalDiscoveryToolbox (CDT)}
    The CausalDiscoveryToolbox supports causal inference in graphs and in pairwise settings. It implements several algorithms, including the PC, LiNGAM, and SAM algorithms.
    https://github.com/FenTechSolutions/CausalDiscoveryToolbox 
    \cite{CDT}
    
    \subsubsection{gCastle}
    GCastle is a causal structure learning toolchain containing various functionalities related to causal learning such as the PC and LiNGAM algorithms. 
    \cite{zhang2021gcastlepythontoolboxcausal}
    
    \section{Results} % Key findings, metrics, or visualizations.

    \subsection{Causal Graph}

    Comparing the various algorithms to discover causal graphs, we found LiNGAM to produce good quality graphs while being significantly faster than many others. 
    
    \subsection{Root Cause}
    
    
    
    \subsection{Visualisation}

    To visualize the root cause, we mainly show all nodes that feed into the target node. This is achieved by inverting the found adjacency matrix and starting from the target node, visit all its descendants recursively. Thus, we get the sub-graph of all nodes influencing the target node. 

    Further, we identify the most influential path according to its total effect on the target node. (TODO?) 
    
    
    \section{Challenges} % Difficulties encountered and solutions.
    The most challenging issue throughout the process was to evaluate possible solutions without knowing the ground truth.

    \section{REMOVE THIS just template}
	
	\begin{figure}
		\centering
		\fbox{\rule[-.5cm]{0cm}{4cm} \rule[-.5cm]{4cm}{0cm}}
		\caption{Sample figure caption.}
	\end{figure}
	
	
	
	
	
	
	\begin{table}
		\caption{Sample table title}
		\label{sample-table}
		\centering
		\begin{tabular}{lll}
			\toprule
			\multicolumn{2}{c}{Part}                   \\
			\cmidrule(r){1-2}
			Name     & Description     & Size ($\mu$m) \\
			\midrule
			Dendrite & Input terminal  & $\sim$100     \\
			Axon     & Output terminal & $\sim$10      \\
			Soma     & Cell body       & up to $10^6$  \\
			\bottomrule
		\end{tabular}
	\end{table}
	
	
	
	
	\printbibliography %Prints bibliography
	%%%%%%%%%%%%%%%%%%%%%%%%%%%%%%%%%%%%%%%%%%%%%%%%%%%%%%%%%%%%
	
	
\end{document}